\chapter{Introdução}
\label{chap:intro}

\section{Enquadramento}
\label{sec:amb} % CADA SECÇÃO DEVE TER UM LABEL
% CADA FIGURA DEVE TER UM LABEL
% CADA TABELA DEVE TER UM LABEL

Neste trabalho foi nos dada a escolha de um projecto de entre 5 temas disponíveis. O tema escolhido pelo grupo foi o  modelo do Sitema Solar. O trabalho foi realizado em C++ com recusro às bibliotecas \ac{OpenGL}, \ac{GLFW}, \ac{GLEW} e \ac{GLM}.

Este relatório foi feito no contexto da unidade curricular de Computação Gráfica da \ac{UBI}. Foi na \ac{UBI} que desenvolvemos todo o trabalho. 

\subsection{Enquadramento Histórico}
O Sistema Solar foi formado à cerca de 4,5 mil milhões de anos a partir de uma densa nuvem de gás interestelar e poeira.

Quando essa nuvem de poeira entrou em colapso, formou uma nebulosa solar (um disco giratório de material).
Vários aglomerados chocaram e formaram objetos cada vez maiores que se juntaram e deram origem aos planetas, planetas anões e luas.

Oficialmente o Sistema Solar tem 8 planetas e 150 luas conhecidas. Mercúrio, Vénus, Terra e Marte são planetas rochosos, Júpiter e Saturno são gigantes gasosos e Urano e Neptuno são gigantes de gelo.

\section{Motivação}
\label{sec:mot}

A Computação Gráfica (CG) é uma área da Ciência da Computação que se dedica
ao estudo e desenvolvimento de técnicas e algoritmos para a síntese de imagens através do computador. Atualmente, é uma das áreas de maior expansão e importância que propicia o desenvolvimento de trabalhos multidisciplinares.

Assim o projeto que se apresenta ajudará a aprofundar o conhecimento nesta área cada vez mais importante e adicionalmente aprofundar os conhecimentos aprendidos na Unidade Curricular

\section{Objetivos}
\label{sec:obj}

O projecto consiste na realização de um modelo interativo do Sistema Solar, em 2D e 3D, onde  projetamos os astros constituintes do mesmo e os seus movimentos (rotações e translações) a uma escala que consideramos adequada, de modo a transmitir ao utilizador uma ideia correta e concreta da realidade.

Caraterísticas da aplicação Gráfica:
\begin{enumerate}
    \item Modelar o sol, os planetas e os seus respetivos satélites;
    \item Texturizar os planetas par aumentar o realismo;
    \item Menus que permitam obter informação sobre os elementos do Sistema Solar;
    \item A câmera pode mover-se para qualquer posição do Sistema Solar, sendo que este movimento afecta a luz que inside sobre os elementos do Sistema Solar.
    \item Utilização do teclado para fazer zoom ao sistema solar e para alterar a vista.
    \item Iluminação através de um foco de luz proveniente do Sol.
    \item Calcular sombras que os planetas e os satélites poderão originar entre si.
\end{enumerate}    

\section{Organização do Documento} -- 
\label{sec:organ}
% !POR EXEMPLO!
De modo a refletir o trabalho que foi feito, este documento encontra-se estruturado da seguinte forma:
\begin{enumerate}
\item O primeiro capítulo -- \textbf{Introdução} -- apresenta o projeto, a motivação para a sua escolha, o enquadramento para o mesmo, os seus objetivos e a respetiva organização do documento.
\item O segundo capítulo -- \textbf{Tecnologias Utilizadas} -- descreve os conceitos mais importantes no âmbito deste projeto, bem como as tecnologias utilizadas durante do desenvolvimento do projeto.
\item O terceiro capítulo -- \textbf{Etapas de Desenvolvimento} -- Neste capítulo são apresentadas quais as etapas tidas em conta no desenevolvimento do projeto e qual o membro do grupo que as realizou.
\item O quarto capítulo -- \textbf{Descrição do funcionamento do Software} -- Nesta etapa explicamos ao pormenor como funcionam as funcões que integram o código, qual o seu intuito e de que forma se relacionam.
\item O quinto capítulo -- \textbf{Considerações Finais} -- Neste capítulo elaborámos uma conclusão sobre o projeto e o conhecimento retido com a sua elaboração.
\item O sexto capítulo -- \textbf{Referências bibliográficas} -- Lista das fontes e referências bibliográficas utilizadas.
\end{enumerate}
