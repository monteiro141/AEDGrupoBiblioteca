\chapter{Conclusões e Trabalho Futuro}
\label{chap:conc-trab-futuro}

\section{Conclusão}
\label{sec:conc-princ}
A concretização do projeto  foi um processo demorado, mas simultanemante estimulante e desafiador para os alunos. A  realização  deste  projeto  constituiu-se  uma  mais  valia  na nossa formação e espelha a sistematização dos conhecimentos teórico-práticos adquiridos ao longo da Unidade Curricular.

Permitiu-nos expandir e consolidar os nossos conhecimentos referentes a C++ e ao OpenGL.

Devido à falta de tempo, não conseguimos implementar o sistema de iluminação como desejávamos, sendo algo a ter em conta em versões futuras do programa.

Um ponto a destacar foi a oportunidade de usar o Blender que facilitou a construção dos objetos.
Foi conseguida a interação do sistema teclado e rato para com o utilizador. As texturas deram um realce mais estético ao nosso Sistema Solar. A implementação de uma funcionalidade no Menu que permite consultar a história de um planeta à escolha, permite ao utilizador aumentar os seus conhecimentos.  

Posto isto e tendo o prazo sido cumprido, achamos que o nosso projeto foi concluido com sucesso.


\section{Trabalho Futuro}
\label{sec:trab-futuro}

No futuro pretendemos aplicar uma melhor iluminação e seria nossa intenção explorar formas de otimizar melhor o código. Seria de salientar a implementação de novas funcionalidades ao programa bem como melhorar a interação com o \textit{user}. 

\begin{thebibliography}{9}
\bibitem{Learn OpenGL} 
\textit{https://learnopengl.com/}

\bibitem{Página da disciplina} 
\textit{https://www.di.ubi.pt/~agomes/cg/}

\bibitem{Solar System Scope} 
\textit{https://www.solarsystemscope.com/textures/}

\bibitem{NASA Solar System Exploration} 
\textit{https://solarsystem.nasa.gov/}

\bibitem{Scale Model of Solar System} 
\textit{https://www.education.com/science-fair/article/scale-model-planets-solar-system/}

\bibitem{Explratorium} 
\textit{https://www.exploratorium.edu/ronh/solar_system/scale.pdf}
\end{thebibliography}