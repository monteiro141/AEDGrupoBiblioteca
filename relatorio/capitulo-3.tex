\chapter{Etapas de desenvolvimmento}
% OU \chapter{Trabalhos Relacionados}
% OU \chapter{Engenharia de Software}
% OU \chapter{Tecnologias e Ferramentas Utilizadas}
\label{chap:tecno-ferra}

\section{Introdução}
\label{chap3:sec:intro}
Neste capítulo são apresentadas quais as etapas tidas em conta no desenevolvimento do projeto e qual o membro do  grupo que as realizou.

\section{Parte técnica}
\label{chap3:sec:...}

\begin{enumerate}
    \item \textbf{Modelação} -- Número de objetos que iremos precisar para fazer o sistema solar definir um tamanho para cada planeta e lua. Pensar como iremos realizar a trajetória dos planetas (1);
    \item \textbf{Blender} -- Fazer no Blender planetas e luas (2);
    \item \textbf{Interação} -- Rato responsável pelo movimento da câmera. Teclado(WASD) responsável pelo movimento pela cena (3);
    \item \textbf{Iluminação} --  Pesquisa do melhor tipo de iluminação a usar e sua aplicação (iluminação flat) (4);
    \item \textbf{Texturização} -- Pesquisar e aplicar as texturas dos planetas e luas (5).
\end{enumerate}

\section{Funcionalidades Extras}

\begin{enumerate}
    \item \textbf{Implementação do Menu} -- Constituido pelas opções FreeWalk e "Planetas", sendo esta última, uma opção que permite ao utilizador escolher um planeta e ver um pouco das suas caraterísticas (6);
    \item \textbf{Background} -- Aplicação de um background do universo ao modelo (7).
\end{enumerate}


\section{Relatório}
\label{chap3:sec:concs}

\begin{enumerate}
    \item \textbf{Pesquisa} -- Pesquisa sobre o funcionamento e aplicação do \LaTeX (8);
    \item \textbf{Motivação do trabalho e tecnologias utilizadas} (9);
    \item \textbf{Explicação do Código} (10);
    \item \textbf{Conclusão e considerações finais} -- Breve conclusão sobre o projeto e pequena sintese sobre os conhecimentos consolidados e/ou adquiridos (11);
    \item \textbf{Finalização do relatório} (12).
\end{enumerate}

\section{Atribuição de Tarefas}

\begin{itemize}
    \item \textbf{Cristiano Santos} --  resolução dos tópicos 1, 3, 5, 6, 7, 8, 9, 10, 12;
    \item \textbf{Bruno Monteiro} -- resolução dos tópicos 1, 2, 4, 5, 6, 8, 9, 10, 12;
    \item \textbf{Alexandre Monteiro} -- resolução  dos tópicos 1, 4, 5, 6, 8, 10, 11, 12.
\end{itemize}

