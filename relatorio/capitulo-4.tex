\chapter{Descrição do funcionamento do Software}
\label{chap:imp-test}

\section{Introdução}
\label{chap4:sec:intro}
Nesta etapa explicámos ao pormenor como funcionam as funcões que integram o código, qual o seu intuito e de que forma se relacionam.

No desenvolvimento do projeto, foi necessário organizar o código em diferentes ficheiros de modo a torná-lo mais organizado e percetível para quem o interpreta.
Sendo assim, temos os segunites ficheiros: \textit{main.cpp}, \textit{draw.hpp} e o  \textit{transferDataToGPU.hpp}. Por sua vez os \textit{shadders} utilizados foram os seguintes: \textit{TransformVertexShader.vertexshader} e o  \textit{TextureFragmentShader.fragmentshader}.

\section{Funcionamento Geral}
Quando o programa é iniciado, o utilizador através das arrow keys up and down, pode selecionar uma das 3 opções disponíveis: "Free Walk", "Planets" e "Quit". Carregando no enter entra numa das opções.

Na opção "Free Walk" o utilizador pode andar livremente pela cena, usando as teclas "WASD" e o rato. Clicando no "R" tem uma perspetiva real do Sistema Solar e clicando no "F" os planetas ficam mais próximos uns dos outros e mais estagnados para permitir ao utilizador ver os planetas de forma mais pormenorizada. Clicando no "Q" volta ao Menu Inicial.

Na opção "Planets" o utilizador pode usar as arrow keys up and down para trocar entre a história dos diferentes planetas. Tal como no menu anterior, pode usar as teclas "WASD" e o rato para se mover na cena. Clicando no "Q" volta ao Menu Inicial.

Na opção "Quit" o programa é encerrado.

\section{\textit{Main.cpp}}
\label{chap4:sec:...}

Iniciamos este ficheiro com a importação das diferentes bibliotecas necessárias:
\begin{lstlisting}[caption=\textit{Includes} das bibliotecas usadas.]
// Include standard headers
#include <stdio.h>
#include <stdlib.h>
#include <time.h>
#include <vector>

// Include GLEW
#include <GL/glew.h>

// Include GLFW
#include <GLFW/glfw3.h>
GLFWwindow* window;

// Include GLM
#include <glm.hpp>
#include <gtc/matrix_transform.hpp>
using namespace glm;

#include "common/shader.hpp"
#include "common/texture.hpp" 
#include "common/controls.hpp" 
#include "common/objloader.hpp"
#include "common/shader.cpp"
#include "common/texture.cpp" 

#include "common/controls.cpp" 
#include "common/objloader.cpp"

#include "transferDataToGPU.hpp"
#include "draw.hpp"
\end{lstlisting}



Criámos uma \textit{typedef struct INFO}, usada para cada um dos seguintes objetos: Background, Sun, Mercury, Venus, Earth, Mars, Jupiter, Saturn, Uranus, Neptune, Ring, Asteroid, Moon, Menu1, Menu2, Menu3, MercuryHistory, VenusHistory, EarthHistory, MarsHistory, JupiterHistory, SaturnHistory, UranusHistory, NeptuneHistory, AsteroidHistory, MoonHistory, SunHistory.
\begin{lstlisting}[caption= \textit{typedef struct INFO}.]
typedef struct {
	GLuint VertexArrayID;
	GLuint MatrixID;
	GLuint Texture;
	GLuint TextureID;
	GLuint uvbuffer;
	GLuint vertexbuffer;
}INFO;
\end{lstlisting}

Criámos uma \textit{typedef struct VECTORS}, usada para cada um dos seguintes objetos: Sphere, Back, VRing, VAsteroid, Menu;
\begin{lstlisting}[caption= \textit{typedef struct VECTORS}.]
typedef struct {
	std::vector<glm::vec3> vertices;
	std::vector<glm::vec2> uvs;
	std::vector<glm::vec3> normals;
}VECTORS;
\end{lstlisting}

Criámos uma \textit{typedef struct TRANSFORMATIONS}, usada para cada um dos seguintes objetos: transfBackground, transfSun, transfMercury, transfVenus, transfEarth, transfMars, transfJupiter, transfSaturn, transfUranus, transfNeptune, transfRing, transfAsteroid, transfMoon, transfList, transfHistory, randomAsteroid[100];
\begin{lstlisting}[caption= \textit{typedef struct TRANSFORMATIONS}.]
typedef struct {
	glm::mat4 self_rotation;
	glm::mat4 center_rotation;
	glm::mat4 translation;
	glm::mat4 scaling;
	float xPosition;
	float yPosition;
	float zPosition;
}TRANSFORMATIONS;
\end{lstlisting}

Criámos a \textit{typedef struct planetRotation}, que foi necessária para a criação da \textit{typedef struct PLANETSROTATIONS};
\begin{lstlisting}[caption= \textit{typedef struct planetRotation}.]
typedef struct {
	float center_Rotation;
	float self_Rotation;
}planetRotation;
\end{lstlisting}

Criámos uma \textit{typedef struct PLANETSROTATIONS}, usada para cada um dos seguintes objetos: UsingRotation;
\begin{lstlisting}[caption= \textit{typedef struct PLANETSROTATIONS}.]
typedef struct {
	planetRotation sun;
	planetRotation mercury;
	planetRotation venus;
	planetRotation earth;
	planetRotation mars;
	planetRotation jupiter;
	planetRotation saturn;
	planetRotation uranus;
	planetRotation neptune;
	planetRotation moon;
	planetRotation asteroidBelt;
}PLANETSROTATIONS;
\end{lstlisting}

Criámos uma \textit{typedef struct DIMENSIONS}, usada para cada um dos seguintes objetos: Real, Fake, Using;
\begin{lstlisting}[caption= \textit{typedef struct DIMENSIONS}.]
typedef struct {
	float sun;
	float mercury;
	float venus;
	float earth;
	float mars;
	float jupiter;
	float saturn;
	float uranus;
	float neptune;
	float moon;
	float asteroidBelt;
}DIMENSIONS;
\end{lstlisting}

Após a criação das \textit{typedef structs}, referenciamos as funções criadas e utilizadas.
São elas as seguintes:
\begin{itemize}
\item \textit{transferDataToGPU()}, contida no \textit{transferDataToGPU.hpp};
\item \textit{draw(), setMVP()}, contida no \textit{draw.hpp};
\item \textit{resetPosPlanetsList()};
\item \textit{selectingMenu1()};
\item \textit{selectingMenu3()};
\item \textit{declareDimensions()};
\item \textit{incrementRotations()};
\item \textit{CelestialBodiesPosition()};
\item \textit{deletebuffers()};
\end{itemize}

\subsection{Funções Implementadas}
\subsubsection{\textit{transferDataToGPU()}}
Começámos por passar como parâmetros a struct INFO, Vectors, a variavel associada aos shaders, o nome da textura e do objeto.

\begin{lstlisting}[caption=\textit{VertexArray} funções.]
glGenVertexArrays(1, (&a->VertexArrayID));
glBindVertexArray((a->VertexArrayID));
\end{lstlisting}
Nestas duas funções vamos buscar o número de Arrays que existem e atribuimo-los ao VertexArrayID da nossa struct INFO.

\begin{lstlisting}[caption=\textit{VertexArray} funções.]
	a->MatrixID = glGetUniformLocation(programID, "MVP");

	a->Texture = loadDDS(textureName);

	a->TextureID = glGetUniformLocation(programID, "myTextureSampler");

	bool res = loadOBJ(objectName, Sphere->vertices, Sphere->uvs, Sphere->normals);
\end{lstlisting}

A primeira função vai utilizar o \textit{MatrixID} para lhe atribuir a localizaçao da matriz MVP dos \textit{shaders}. A segunda vai atribuir a variável textura à nossa textura do que nos é passada em parametro para uma função de carregar texturas (loadDDS(\textit{texturename()}). A terceira textura, no fundo, faz o mesmo que a priemira função mas aplicada à textura. A quarta e ultima, vai atribuir à nossa \textit{struct VECTORS} os vértices, os UVs e as normais do objecto.


\begin{lstlisting}[caption=\textit{Buffers} funções.]
glGenBuffers(1, &a->vertexbuffer);
glBindBuffer(GL_ARRAY_BUFFER, a->vertexbuffer);
glBufferData(GL_ARRAY_BUFFER, (&Sphere->vertices)->size() * sizeof(glm::vec3), &Sphere->vertices[0][0], GL_STATIC_DRAW);

glGenBuffers(1, &a->uvbuffer);
glBindBuffer(GL_ARRAY_BUFFER, a->uvbuffer);
glBufferData(GL_ARRAY_BUFFER, (&Sphere->uvs)->size() * sizeof(glm::vec2), &Sphere->uvs[0][0], GL_STATIC_DRAW);
\end{lstlisting}

\subsubsection{\textit{draw(), setMVP()}}
\begin{lstlisting}[caption=\textit{setMVP()}.]
glm::mat4 setMVP(TRANSFORMATIONS b) {
	// Compute the MVP matrix from keyboard and mouse input
	computeMatricesFromInputs();
	glm::mat4 ProjectionMatrix = getProjectionMatrix();
	glm::mat4 ViewMatrix = getViewMatrix();
	glm::mat4 ModelMatrix = glm::mat4(1.0);
	glm::mat4 MVP = ProjectionMatrix * ViewMatrix * ModelMatrix * b.center_rotation * b.translation * b.self_rotation * b.scaling;
	return MVP;
}
\end{lstlisting}
Esta função vai receber como parâmetro a \textit{struct Transformations}, que vai ser usada para aplicar à matriz MVP (\textit{computeMatricesFromInputs()}). A matriz MVP foi calculada da seguinte forma : \textit{MVP = ProjectionMatrix * ViewMatrix * ModelMatrix * b.center\_rotation * b.translation * b.self\_rotation * b.scaling}.

\begin{lstlisting}[caption=\textit{draw()}.]
void draw(INFO *a, TRANSFORMATIONS b, GLuint programID, GLuint primitive, std::vector<glm::vec3>* verticesName) {
	// Use our shader
	
	glUseProgram(programID);
	glm::mat4 MVP= setMVP(b);
	// Send our transformation to the currently bound shader, 
	// in the "MVP" uniform
	glUniformMatrix4fv(a->MatrixID, 1, GL_FALSE, &MVP[0][0]);

	// Bind our texture in Texture Unit 0
	glActiveTexture(GL_TEXTURE0);
	glBindTexture(GL_TEXTURE_2D, a->Texture);
	// Set our "myTextureSampler" sampler to use Texture Unit 0
	glUniform1i(a->TextureID, 0);

	// 1rst attribute buffer : vertices
	glEnableVertexAttribArray(0);
	glBindBuffer(GL_ARRAY_BUFFER, a->vertexbuffer);
	glVertexAttribPointer(
		0,                  // attribute
		3,                  // size
		GL_FLOAT,           // type
		GL_FALSE,           // normalized?
		0,                  // stride
		(void*)0            // array buffer offset
	);

	// 2nd attribute buffer : UVs
	glEnableVertexAttribArray(1);
	glBindBuffer(GL_ARRAY_BUFFER, a->uvbuffer);
	glVertexAttribPointer(
		1,                                // attribute
		2,                                // size
		GL_FLOAT,                         // type
		GL_FALSE,                         // normalized?
		0,                                // stride
		(void*)0                          // array buffer offset
	);

	// Draw the triangle !
	glDrawArrays(primitive, 0, verticesName->size());

	glDisableVertexAttribArray(0);
	glDisableVertexAttribArray(1);
}
\end{lstlisting}

A função \textit{draw()} recebe como parâmetros as structs INFO e TRANSFORMATIONS, o \textit{programID} (link para os shaders), a \textit{primitive} e o \textit{verticesName} (vetor com os vertices).
A \textit{glActiveTexture} e a \textit{glBindTexture} são usadas para carregar as texturas.

A primeira ocorrência destas funções, \textit{glEnableVertexAttribArray}, \textit{glBindBuffer}, \textit{glVertexAttribPointer} envia os vértices deste objeto para os shaders.
A segunda, envia para os shaders os UVs, que por si são as coordenadas das texturas.

Posteriormente ele vai desenhar de acordo com a primitiva passada como parametro, desde o índice 0 até ao tamanho do vetor.


\subsubsection{\textit{resetPosPlanetsList()}}
Esta função vai dar \textit{reset} à posição da camâra quando o utilizador alterna entre planetas no Menu \textit{Planets}
\begin{lstlisting}[caption=\textit{resetPosPlanetsList()}.]
void resetPosPlanetsList() {
	position = glm::vec3(145.78f, 2.13f, 39.61f);
	horizontalAngle = -1.55f;
	verticalAngle= -0.03f;
\end{lstlisting}

\subsubsection{\textit{selectingMenu1()}}
Esta função é responsável pela escolha da opção do Menu Inicial.
\begin{lstlisting}[caption=\textit{ Pequeno excerto da função selectingMenu1()}.]
	if (glfwGetKey(window, GLFW_KEY_ENTER) == GLFW_PRESS && glfwGetKey(window, GLFW_KEY_ENTER) == GLFW_RELEASE)
	{
		if (selectedMenu1 == 1)
			currentMenu = 2;
		else if (selectedMenu1 == 2) {
			resetPosPlanetsList();
			currentMenu = 3;
		}

		else if (selectedMenu1 == 3)
			currentMenu = 4;
	}
\end{lstlisting}


\subsubsection{\textit{selectingMenu3()}}
Esta função é a responsável pela escolha da história dos planetas que irão aparecer.
\begin{lstlisting}[caption=\textit{ Pequeno excerto da função selectingMenu3()}.]
draw(&Background, transfBackground, programID, GL_TRIANGLES, &Back.vertices);
	if (glfwGetKey(window, GLFW_KEY_DOWN) == GLFW_PRESS && glfwGetKey(window, GLFW_KEY_DOWN) == GLFW_RELEASE)
	{
		resetPosPlanetsList();
		if (selectedMenu3 == 11)
			selectedMenu3 = 1;
		else
			selectedMenu3++;
	}
	if (glfwGetKey(window, GLFW_KEY_UP) == GLFW_PRESS && glfwGetKey(window, GLFW_KEY_UP) == GLFW_RELEASE)
	{
		resetPosPlanetsList();
		if (selectedMenu3 == 1)
			selectedMenu3 = 11;
		else
			selectedMenu3--;
	}
\end{lstlisting}

\subsubsection{\textit{declareDimensions()}}
Nesta função nós declaramos a escala real dos planetas e uma escala \textit{Fake}, sendo esta última usada para visualizar melhor os planetas. Atribuimos um valor \textit{random} ã posição inicial dos planetas em torno do Sol.
\newpage
\begin{lstlisting}[caption=\textit{Pequeno excerto da função declareDimensions()}.]
Real.sun = 0.0f * AU;
Real.mercury = 0.28f * AU;
Real.venus = 0.367f * AU;

(...)

Fake.sun = 0.0f * AU;
Fake.mercury = 0.28f * AU;
Fake.venus = 0.367f * AU;

(...)

UsingRotation.sun.center_Rotation = (float)(rand() % 365);
UsingRotation.mercury.center_Rotation = (float)(rand() % 365);
UsingRotation.venus.center_Rotation = (float)(rand() % 365);
\end{lstlisting}

\subsubsection{\textit{incrementRotations()}}
Esta função é a responsável pela escala de rotação e translação dso planetas.
\begin{lstlisting}[caption=\textit{ Pequeno excerto da função incrementRotations()}.]
UsingRotation.sun.center_Rotation += 0.0f;
UsingRotation.mercury.center_Rotation += 4.20f * rotationScale;
UsingRotation.venus.center_Rotation += 1.43f * rotationScale;

(...)

UsingRotation.sun.self_Rotation += 0.0115f;
UsingRotation.mercury.self_Rotation += 0.017f;
UsingRotation.venus.self_Rotation -= 0.004f;
\end{lstlisting}

\subsubsection{\textit{CelestialBodiesPosition()}}
Esta função define a translação em torno do Sol, a rotação sobre si mesma e o tamanho do planeta(\textit{scaling}).

\begin{lstlisting}[caption=\textit{ Pequeno excerto da função CelestialBodiesPosition()}.]
//Sol
	transfSun.center_rotation = glm::rotate(glm::mat4(1.0), glm::radians(UsingRotation.sun.center_Rotation), glm::vec3(0, 1, 0));
	transfSun.self_rotation = glm::rotate(glm::mat4(1.0), glm::radians(UsingRotation.sun.self_Rotation), glm::vec3(0, 1, 0));
	transfSun.translation = glm::translate(glm::mat4(1.0), glm::vec3(Using.sun, 0.0f, 0.0f));
	transfSun.scaling = glm::scale(glm::mat4(1.0), glm::vec3(19.0f));

//Mercury
	transfMercury.center_rotation = glm::rotate(glm::mat4(1.0), glm::radians(UsingRotation.mercury.center_Rotation), glm::vec3(0, 1, 0));
	transfMercury.self_rotation = glm::rotate(glm::mat4(1.0), glm::radians(UsingRotation.mercury.self_Rotation), glm::vec3(0, 1, 0));
	transfMercury.translation = glm::translate(glm::mat4(1.0), glm::vec3(Using.mercury, 0.0f, 0.0f));
\end{lstlisting}

\subsubsection{\textit{deletebuffers()}}
Nesta função apagamos o \textit{programID}, todas as texturas, \textit{buffers}(contêm vértices dos objectos e UVs). Apagamos ainda os \textit{VertexArrayID}'s.
\begin{lstlisting}[caption=\textit{ Pequeno excerto da função deletebuffers()}.]
glDeleteProgram(programID);

//Delete Textures
glDeleteTextures(1, &Background.Texture);
glDeleteTextures(1, &Sun.Texture);

//Delete buffers
glDeleteBuffers(1, &Background.vertexbuffer);
glDeleteBuffers(1, &Background.uvbuffer);

//Delete VertexArrays
glDeleteVertexArrays(1, &Background.VertexArrayID);
glDeleteVertexArrays(1, &Sun.VertexArrayID);
\end{lstlisting}

\subsubsection{\textit{Cintura de Asteróides}}
Depois de declarmos o randomAsteroid[100]  do tipo \textit{strut} TRANSFORMATIONS, percorremos cada posição do Array e atribuímos uma posição para o asteróide correspondente e um tamanho variável. De forma a tornar isto o mais \textit{random} possível gerámos uma nova \textit{seed} em cada execução do programa.

\begin{lstlisting}[caption=\textit{ Excerto de código para a criação da cintura de Asteroides}.]
    int j = 0;
    float i = 0;
    //Random seed
    time_t t;
    srand((unsigned) time(&t));
    for (i = 0, j = 0; i < 360; i += 3.6, j++)
    {
        //float cent_Rotation = (((float)(rand()) / RAND_MAX) * i) + i;
        float cent_Rotation = (rand() % 4) + i - 0.4;
        float transl = ((rand() % 10) + AU - 10);
        float scal = ((float)(rand()) / RAND_MAX) * 0.1f;

        randomAsteroid[j].self_rotation = glm::rotate(glm::mat4(1.0), glm::radians(1.0f), glm::vec3(0, 1, 0));
        randomAsteroid[j].center_rotation = glm::rotate(glm::mat4(1.0), glm::radians(cent_Rotation), glm::vec3(0, 1, 0));
        randomAsteroid[j].translation = glm::translate(glm::mat4(1.0), glm::vec3(transl, 0.0f, 0.0f));
        randomAsteroid[j].scaling = glm::scale(glm::mat4(1.0), glm::vec3(scal));
    }
\end{lstlisting}

\begin{figure}[H]
    \centering
    \includegraphics[width=12cm]{Images/cinturaasteroides.png}
    \caption{Cintura de Asteróides vista de cima}
    \label{fig:label}
\end{figure}


\begin{figure}[H]
    \centering
    \includegraphics[width=12cm]{Images/Menu.png}
    \caption{Menu Inicial}
    \label{fig:label}
\end{figure}

\begin{figure}[H]
    \centering
    \includegraphics[width=12cm]{Images/SolarSystemPerspective.png}
    \caption{Perspetiva do Sistema Solar}
    \label{fig:label}
\end{figure}

\begin{figure}[H]
    \centering
    \includegraphics[width=12cm]{Images/earth_history.png}
    \caption{História da Terra}
    \label{fig:label}
\end{figure}

\section{\textit{Shaders}}
\subsection{\textit{VertexShader}}
Os \textit{shaders} recebem as posições dos vértices do objeto que são armazenadas no \textit{vertexPosition\_modelspace}. Recebem também as posições dos objetos no qual terá a textura, estas são guardadas no \textit{vertexUV}.

O MVP é uma matriz 4x4 recebida do \textit{draw()} através do: \newline \textit{glUniformMatrix4fv(a->MatrixID, 1, GL\_FALSE, \&MVP[0][0])}. 

Primeiramente transformamos o \textit{vertexPosition\_modelspace} num vec4, uma vez que a matriz MVP é um mat4. Desta forma a multiplicação é possível para obter o \textit{gl\_Position}.

O UV recebe os \textit{VertexUV} e envia-os para o \textit{FragmentShader}.

\begin{lstlisting}[caption=\textit{VertexShader}]
#version 330 core

layout(location = 0) in vec3 vertexPosition_modelspace;
layout(location = 1) in vec2 vertexUV;

out vec2 UV;

uniform mat4 MVP;

void main(){

	gl_Position =  MVP * vec4(vertexPosition_modelspace,1);
	
	UV = vertexUV;
}
\end{lstlisting}

\subsection{\textit{FragmentShader}}
O \textit{FragmentShader} recebe os UV's vindos do \textit{VertexShader} e a textura (\textit{myTextureSampler}) recebida do \textit{transferDataToGPU()} através da função:\newline
a->TextureID = glGetUniformLocation(programID, "myTextureSampler").

Aplicamos a função \textit{texture()} onde passamos como argumento a textura e os UV's. Esta tranformação é enviada como vec3 para o OpenGL.

Como só existe um único \textit{out} o OpenGL já sabe qual é a variável das cores.

\begin{lstlisting}[caption=\textit{FragmentShader}]
#version 330 core

in vec2 UV;

out vec3 color;

uniform sampler2D myTextureSampler;

void main(){

	color = texture( myTextureSampler, UV ).rgb;
}
\end{lstlisting}




