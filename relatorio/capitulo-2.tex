\chapter{Tecnologias Utilizadas}
% OU \chapter{Trabalhos Relacionados}
% OU \chapter{Engenharia de Software}
% OU \chapter{Tecnologias e Ferramentas Utilizadas}
\label{chap:estado-da-arte}

\section{Introdução}
\label{chap2:sec:intro}
Neste capítulo descrevemos  os  conceitos mais  importantes  no âmbito  deste  projeto,  bem  como  as  tecnologias utilizadas durante do desenvolvimento da projeto. 

\section{Conceitos importantes}
\label{chap2:sec:citacoes}

Alguns dos conceitos mais importantes:
\begin{enumerate}
    \item \textbf{Básicos de Geometria} -- Conhecimmentos teóricos sobre vetores, matrizes e algebra;
    \item \textbf{Transformações Geométricas} -- Noções básicas de transformações de matrizes ( matrizes de tranformação, rotação e \textit{scaling});
    \item \textbf{Windows and Viewports} -- Uma viewport define em coordenadas normalizadas uma área retangular no display onde a imagem dos dados aparece. Uma window define a área retangular em coordenadas.
    \item \textbf{Shading and Ilumination} -- O shading é a implementação do modelo de iluminação nos pixeis e superfícies poligonais dos objetos gráficos. Por sua vez, a Ilumination é a forma como o shading é aplicado ns diferentes objetos.
    \item \textbf{Color} -- Atribuição de diferentes cores da escala RGB (em OpenGL, entre 0 e 1) aos diversos pixeis e objetos.
\end{enumerate}

\section{Tecnologias Utilizadas}
\label{chap2:sec:...}
O C++ é uma linguagem de programação e de uso geral. É uma das linguagens mais populares desde 1990 e foi criada por Bjarne Stroustrup.

O OpenGL é uma \ac{API} livre utilizada na computação gráfica para o desenvolvimento de aplicações gráficas, ambientes 3D, jogos, etc.

O GLM generaliza a regressão linear permitindo que este seja relacionado à variável de resposta por meio de uma função de ligação, permitindo que a magnitude da variância de cada medição seja uma função do seu valor previsto.

O GLFW é uma biblioteca que pode ser usada com OpenGL. Esta permite que programadores possam criar e manusear janelas e contextos OpenGL. Permite também interagir com joysticks, rato e teclado.

O GLEW é uma biblioteca de carregamento de extensões C/C++ de código \textit{open-source}. Este fornece mecanismos eficientes para determinar quais as extensões OpenGL que são suportadas na plantaforma alvo.

O Visual Studio \ac{IDE} é uma plataforma de lançamento criativa que se pode usar para alterar, depurar, construir código e publicar um aplicação.
